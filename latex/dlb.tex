\chapter{Analysis of the Design Load Basis}

\section{Introduction}

The design load basis (DLB) considered for this report is based on the IEC 61400-1 standard. A more elaborate discussion and interpration of that standard is given in \cite{hansen_design_2015}. The aeroelastic code used for these investigations is HAWC2 \cite{hawc2_manual}.

The following rotor configurations have been considered:
\begin{itemize}
	\item reference (referred to as baseline or baseline100kw in the figures)
	\item KB5
\end{itemize}

The statistics are included in section \ref{tables:baseline-vs-KB6} as tables, and as plots in section \ref{app:baseline-vs-KB6}. The plots show the minima, mean, maxima and standard deviations for all of the simulations considered within the given design load case. The tables give the mininima of the minima, and the maxima of the maxima for each DLC and channel. For a more elaborate description of the considered design load basis the reader is referred to \cite{hansen_design_2015}. The considered design load cases (DLC's) for this report are the following: DLC's 1.2, 2.4, 3.1, 4.1, and 6.4.

This report does not include selected time series plots of the design load basis analysis, although some of the conclusions presented within the the discussion section (see section \ref{sec:dlb:discussion}) are based on the analysis of individual time series.

Appendix \ref{app:baseline-vs-KB6} include the comparison between the reference rotor and KB5. The following channels are considered:
\begin{itemize}
	\item Electrical power [W], excluding losses.
    \item Rotor speed [rpm]
    \item Blade 1, 2 and 3 pitch angles [deg]
    \item Controller status flag [0-6], 0: normal operation, 1: shut down due to over speed
    \item Tower base for-aft bending moment [kNm]
    \item Tower base side-side bending moment [kNm]
    \item Yawing moment tower top [kNm]
    \item Shaft torsion moment [kNm]
    \item Blade root 1, 2 and 3 flap-wise bending moments [kNm] (pitching coordinates)
    \item Blade root 1, 2 and 3 edge-wise bending moments [kNm] (pitching coordinates)
    \item Blade root 1, 2 and 3 torsion bending moments [kNm] (pitching coordinates)
    \item Minimum tower to blade tip distance [m]
\end{itemize}

To accommodate the comparison the figures include two different rotors, but the wind speed or wind direction is slightly off-set from the actual value. For example, for each wind speed of 10 m/s, case 1 is plotted at a  value slightly below 10 m/s, and case 2 at a value slightly above. Both cases have been run at the same wind speed of 10 m/s.


\section{Stiff Tower}
\label{sec:dlb:Stiff Tower}

The tower stiffness was first estimated based on a quantitative description and drawings provided by the turbine manufacturer. However, the stiffness and mass properties of the derived structural model ended up being very close to the 3P frequency at rated wind speed and above. This resulted in a sharp increase in tower side-side, blade-edge and shaft torsional loads. Considering that this platform has been produced and is being operated in the field, it is assumed that tower model initially derived from the given data was not sufficiently accurate. Until more detailed structural data becomes available, the tower is considered to be stiff. Once a sufficiently accurate description of the tower is known a flexible tower model can be introduced to re-assess the design load basis. Depending on the eigenfrequencies and damping of the tower, the introduction of tower flexibility is expected to affect the results.


\section{Controller Tuning}
\label{sec:dlb:Controller Tuning}

For this study the basic DTU Wind Energy controller is used \cite{hansen_basic_2013}. The controller is tuned based on the reference rotor using the pole placement technique as implemented for a 1 DOF model in HAWCStab2.

The controller tuning settings are assumed to be fixed for both the baseline and KB5 rotor since the controller tuning will not modifiable for the prototype with KB5 blades. Note that the design procedure did not include any additional constraints to mitigate the effects of a fixed controller tuning setup. Under normal circumstances, a re-designed rotor requires different controller tuning parameters compared to the reference case. The effect of not re-tuning the controller for KB5 is not considered within the scope of this report.

In addition to the considered design load cases, one extra load case is included: normal operating conditions with wind shear and tower shadow, but without any turbulence. Additionally, a 1 m/s wind step is added to the wind speed towards the end of the simulation. The purpose of this load case is to asses steady power curve performance. This test case is labelled as "test\_steadysteps".


\section{Discussion and Conclusion}
\label{sec:dlb:discussion}

The cases without turbulence (see section \ref{sec:baseline-vs-KB6:test_steadysteps} show that there is indeed an increased power output for KB5 in below rated conditions while at the same time the loads are either similar or reduced compared to the baseline rotor. This trend is also visible for the other load cases (see sections \ref{sec:baseline-vs-KB6:dlc12}-\ref{sec:baseline-vs-KB6:dlc64}). There is one outlier for KB5 that results in an increased tower base for-aft bending moment in DLC1.2 (see figure \ref{fig:baseline-vs-KB6:dlc12:tower-base-fa}). The cause of this could be due to the sub-optimal controller tuning of KB5, but a more detailed follow-up investigation is required before a final conclusion can be rendered. Further, a load increase can be observed for the blade root torsional bending moment. This increase is a consequence of the swept blade geometry.


\section{Future Work}
\label{sec:dlb:Future Work}

Considering that KB5 (or a rotor similar to KB5) is forecasted to be build and operated as a prototype, the following points are suggested for further analysis:

\begin{itemize}
	\item Include time series analysis of certain indicative results.
	\item Determine an accurate model for the tower (eigenfreqencies, structural damping).
	\item Determine controller tuning parameters from the turbine platform.
	\item Sensitivity study regarding controller tuning parameters to account for the uncertainty caused by the unknown actual controller tuning parameters of the platform.
	\item Tabulated results showing the differences in load statistics.
	\item Tabulated extreme loads observed within each of the DLB's.
	\item Brake down of fatigue loads, and life time fatigue loads.
\end{itemize}
